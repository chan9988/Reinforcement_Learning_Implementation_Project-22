\section{Empirical Evaluation}
\label{section:evaluation}



% =======================================================================

\subsection{Simple MDP and Random MDPs}

The author of this paper construct a handcrafted simple MDP, as shown in figure\ \ref{fig:simple_mdp}.
It is composed of two states $s_1$ and $s_2$. State $s_2$ is a terminal state, and only $s_1$ is a non-terminal state.
The edges are labeled with a transition probability (unsigned) and a reward number (signed). Discount factor $\gamma=0.98$.
State $s_1$ has two actions a and b. In the following simple MDP experiments, we mainly focus on the Q value of a and b.

% {\begin{wrapfigure}{l}{0.25\textwidth}
%     \centering    
%     \includegraphics[width=0.25\textwidth]{simple_mdp_fig.png}
%     \caption{Simple MDP Settings}\label{fig:simple_mdp}
% \end{wrapfigure}}

\begin{figure}[H]
    \centering
    \includegraphics[width=0.25\textwidth]{simple_mdp_fig.png}
    \caption{Simple MDP}\label{fig:simple_mdp}
\end{figure}

% =======================================================================
\subsubsection{Simple MDP\ -\ SARSA}

Figure\ \ref{fig:simple_mdp_sarsa} shows the results of SARSA (algorithm) running 2000 episodes in this MDP. 
The blue and green lines depict the Q-values for both actions in each episode. 
Boltzmann curves have significantly larger oscillations. 
Its swing range is around 0.3 to 0.4, while the Q value of the Mellowmax curve is
relatively stable; its swing range is around 0.2.

{\centering
\begin{figure}[H]
\begin{tabular}{cc}
\begin{subfigure}{0.45\textwidth}\centering\includegraphics[width=0.9\columnwidth, height=5cm]{SARSA_Boltz_paper.png}\caption{Boltz (paper)}\end{subfigure}&
\begin{subfigure}{0.45\textwidth}\centering\includegraphics[width=0.9\columnwidth, height=5cm]{SARSA_sampleMDP_Boltzmann Softmax_smoothed.png}\caption{Boltz (replication)}\end{subfigure}\\
\newline
\begin{subfigure}{0.45\textwidth}\centering\includegraphics[width=0.9\columnwidth, height=5cm]{white.png}\caption{Mellowmax (paper does not has this)}\end{subfigure}&
\begin{subfigure}{0.45\textwidth}\centering\includegraphics[width=0.9\columnwidth, height=5cm]{SARSA_sampleMDP_Mellowmax_smoothed.png}\caption{Mellowmax (our)}\end{subfigure}\\
\end{tabular}
\caption{SARSA with bolz/mm policy ($\beta=16.55$, $\omega=16.55$)}\label{fig:simple_mdp_sarsa}
\end{figure}}

% =======================================================================
\subsubsection{Simple MDP\ -\ GVI}

Compared with value iteration, GVI doesn not restrict the way to compute 
$\otimes Q(s',\cdot)$. Max, Boltzmann Softmax, Mellowmax, and other operators can be used as an operator.
(For more detail of GVI, please refer to\cite{littman1996generalized}.)

Figure\ \ref{fig:simple_mdp_gvi_paper} (from paper) demonstrates the GVI result of
Boltzmann Softmax and Mellowmax. An arrow is the updating direction and step size of a Q pair.
Black points are the convergence fixed points. As we can see, in this case, there are two convergence points of Boltzmann Softmax's,
while Mellowmax has only one convergence point. 

Besides, the arrows on the figure tell us that the updating size is small 
when the Q pair locates at the position near the convergence point.
This phenomenon is more obvious at the position between two convergence points especially.
Thus, when there are more than one fixed points, 
the value may stay in the area between two convergence points with a tiny updating size, 
needing more iterations to reach a convergence point.

Boltzmann Softmax does not have the non-expansion property, 
so it is not under the convergence guarantee from the original GVI paper (\cite{littman1996generalized}). 
If there are > 1 convergence fixed points, the noise may drive it to swing 
between different points and lead to instability.

{\centering
\begin{figure}[H]
\begin{tabular}{cc}
\begin{subfigure}{0.45\textwidth}\centering\includegraphics[width=0.95\columnwidth, height=5cm]{simpleMDP_GVI_boltz_paper.png}\caption{Boltz (paper)}\end{subfigure}&
\begin{subfigure}{0.45\textwidth}\centering\includegraphics[width=0.95\columnwidth, height=5cm]{simpleMDP_GVI_mm_paper.png}\caption{Mellowmax (paper)}\end{subfigure}\\
\end{tabular}
\caption{Paper's GVI with bolz/mm policy ($\beta=16.55$, $\omega=16.55$)}\label{fig:simple_mdp_gvi_paper}
\end{figure}}


Figure\ \ref{fig:simple_mdp_gvi_our_bolz} is our replication of GVI with Boltzmann Softmax policy under different hyper-parameter $\beta$.
A blue arrow displays the first updating direction and relative size  of one initial point.
A green arrow displays the direction and relative updating size from starting point to convergence point.
A red/balck point is a convergence fixed point.
A range of $\beta$ is tried, and we find under some $\beta$, there more than one convergence points as the figures shown.

{\centering
\begin{figure}[H]
\begin{tabular}{ccc}
\begin{subfigure}{0.33\textwidth}\centering\includegraphics[width=0.9\columnwidth, height=4.5cm]{our_simple_boltz1.png}\caption{$\beta=16.7$}\end{subfigure}&
\begin{subfigure}{0.33\textwidth}\centering\includegraphics[width=0.9\columnwidth, height=4.5cm]{our_simple_boltz2.png}\caption{$\beta=16.92$}\end{subfigure}&
\begin{subfigure}{0.33\textwidth}\centering\includegraphics[width=0.9\columnwidth, height=4.5cm]{our_simple_boltz3.png}\caption{$\beta=17.06$}\end{subfigure}\\
\end{tabular}
\caption{Our Boltzmann GVI replication (boltz)}\label{fig:simple_mdp_gvi_our_bolz}
\end{figure}}

In conparison, Mellowmax does not show that it has more than one fixed points in our experiment.
It's relative stable. We try to change $\omega$ from around 16.7 to 17, and 
the fixed point only move for a very small distance and stay near the point, as figure\ \ref{fig:simple_mdp_gvi_our_mm} displays.

{\centering
\begin{figure}[H]
\begin{tabular}{ccc}
\begin{subfigure}{0.33\textwidth}\centering\includegraphics[width=0.9\columnwidth, height=4.5cm]{our_simple_mdp_gvi_mm167.png}\caption{$\beta=16.7$}\end{subfigure}&
\begin{subfigure}{0.33\textwidth}\centering\includegraphics[width=0.9\columnwidth, height=4.5cm]{our_simple_mdp_gvi_mm1692.png}\caption{$\beta=16.92$}\end{subfigure}&
\begin{subfigure}{0.33\textwidth}\centering\includegraphics[width=0.9\columnwidth, height=4.5cm]{our_simple_mdp_gvi_mm176.png}\caption{$\beta=17.06$}\end{subfigure}\\
\end{tabular}
\caption{Our Boltzmann GVI replication (mm)}\label{fig:simple_mdp_gvi_our_mm}
\end{figure}}

The exact value of beta when it does not converge at only one point 
derived by us is not precisely equivalent to that shown on paper. 
But both of it show GVI with Boltzmann Softmax under a period of $\beta$ 
near 16.7 converge at more than one points.


The following figure is from paper and it is obtained by trying different betas for GVI using the Boltzmann Softmax policy. 
Similar to the results in Figure\ \ref{fig:simple_mdp_gvi_paper} and\ \ref{fig:simple_mdp_gvi_our_bolz},
there are more than one convergence fixed point under some $\beta$.

\begin{figure}[H]
    \centering
    \includegraphics[width=0.7\textwidth]{simpleMDP_boltz_paper_diff_beta.png}
    \caption{Number of different fixed points under different $\beta$ (GVI with boltz)}\label{fig:simple_mdp_boltz_paper_diff_beta}
\end{figure}

Figure\ \ref{fig:simple_mdp_gvi_our_diff_beta} is our result. 
Subfigure\ \ref{subfig:our_diff_beta_boltz} shows fixed points of GVI using Boltzmann Softmax 
under different $\beta$. Although the precise value is not completely the same as figure\ \ref{fig:simple_mdp_boltz_paper_diff_beta},
it demonstrates the same result as figure\ \ref{fig:simple_mdp_boltz_paper_diff_beta}-there are more than one convergence fixed points under some $\beta$.

We make an additional subfigure\ \ref{subfig:our_diff_beta_mm} to show Mellowmax's result. 
Akin to what figure 5 shows, the fixed point of Mellowmax doesn't change a lot under different hyper-parameter $\omega$.
By comparison, it can be found that the convergence point of Boltzmann Softmax is more sensitive to hyperparameter tuning, 
while the convergence point of Mellomax is less affected by hyperparameter changes.

\begin{figure}[H]
\centering
\begin{tabular}{cc}
\begin{subfigure}{0.45\textwidth}\centering\includegraphics[width=0.9\columnwidth]{simpleMDP_boltz_our_diff_beta.png}\caption{Boltzmann Softmax}\label{subfig:our_diff_beta_boltz}\end{subfigure}&
\begin{subfigure}{0.45\textwidth}\centering\includegraphics[width=0.9\columnwidth]{simpleMDP_mm_our_diff_beta.png}\caption{Mellowmax}\label{subfig:our_diff_beta_mm}\end{subfigure}\\
\end{tabular}
\caption{Our replication}\label{fig:simple_mdp_gvi_our_diff_beta}
\end{figure}



% =======================================================================
\subsubsection{Random MDPs}

The experiments discussed so far were run in a handcrafted MDP environment. 
To see if these properties also exist under natural conditions, 
the paper conducts experiments on random MDPs to observe if 
randomly generated MDPs using these two policies for GVI also have these properties.

Figure\ \ref{fig:randomMDP_paper} is the result from paper. 
It tries 200 MDPs. No termination means the GVI cannot converge within a predefined max number
of iterations. Boltzman Softmax has these results: non-termination and >1 fixed points, 
while Mellowmax does not. In addition, Boltzmann Softmax requires more iterations on average to converge.

\begin{figure}[H]
    \centering
    \includegraphics[width=0.5\textwidth]{randomMDP_paper.png}
    \caption{Random MDPs (paper). Max iteration is set 1000.}\label{fig:randomMDP_paper}
\end{figure}

Figure\ \ref{fig:randomMDP_paper} is our result.
The way we construct random MDPs is a little different from the paper. 
See Algorithm\ \ref{alg:RandomMDP-GVI} for details. We tried 200 MDPs; 100 trials were run for each MDP. 
The values of no-termination and average iteration in the figure are the results of averaging all MDPs' all trials.
The values of > 1 fixed points are the results of averaging all MDPs' number.
It can be seen that it presents the same characteristics as in Figure\ \ref{fig:randomMDP_paper}.

\begin{figure}[H]
    \centering
    \begin{tabular}{ |c|c|c|c| } 
     \hline
     { }  & avg \# no terminate & avg \# > 1 fixed points & avg interation\\ 
     \hline
     $\text{boltz}_\beta$ & 0.00675 & 0.03 & 1085.0973\\ 
     \hline
     $\text{mm}_\omega$ & 0 & 0 & 1048.794 \\ 
     \hline
    \end{tabular}
    \caption{Random MDPs (our). Max iteration is set 2000. There are 200 MDPs tried; each MDP has 100 trials.}\label{fig:randomMDP_our}
\end{figure}


\subsubsection{How to determin wheter > 1 fixed points?}
{Note that the paper doesn't explain how they determine whether > 1 fixed points. 
The approach we use to decide if > 1 fixed points is to sample different initial Q values 
in each trial to see if there are different fixed points in all trials. 
That's why we run 100 trials for each MDP in figure \ref{fig:randomMDP_our}. 
Although getting no > 1 fixed points by this method doesn't means it must actually have no > 1 fixed points, 
at least, we can derive it has a higher probability or less probability to have > 1 fixed points. 
For example, in our experiment result, Mellowmax never has > 1 fixed points among all trials, 
while Boltzmann has. We can infer that Boltzmann has at least a higher probability to have > 1 fixed points than Mellowmax.
}\label{randomMDP_finding_fixed_points_issue}


% ###########################################################################
% ###########################################################################

\subsection{Taxi Domain}

Sarsa with epsilon-greedy, learning rate $\alpha = 0.1$, discount factor $\gamma = 1 $, we test different $\epsilon$ values, $\epsilon = 0.05, 0.1, 0.2, 0.3, 0.5$, and the result are averaged over 6 independent runs, each consisting of 100000 time steps.

Sarsa with Boltzmann softmax, learning rate $\alpha = 0.1$, discount factor $\gamma = 1 $, we test different $\beta$ values, $\beta = 0.5, 1, 2, 3, 5, 10$, and the result are averaged over 6 independent runs, each consisting of 100000 time steps.

Sarsa with Mellowmax softmax, learning rate $\alpha = 0.1$, discount factor $\gamma = 1 $, we test different $\omega$ values, $\omega = 0.5, 1, 2, 3, 5, 50$, and the result are averaged over 6 independent runs, each consisting of 10000 time steps. We know that when $\omega \to \infty$, Mellowmax will behave like max operator, and we set $\omega$ to 50 so that the result will similar to the paper and show that phenomenon.

{\centering
\begin{figure}[H]
\begin{tabular}{ccc}
\begin{subfigure}{0.32\textwidth}\centering\includegraphics[width=0.98\columnwidth, height=4cm]{taxi_1.png}\caption{Boltz (paper)}\end{subfigure}&
\begin{subfigure}{0.32\textwidth}\centering\includegraphics[width=0.98\columnwidth, height=4cm]{taxi_2.png}\caption{Boltz (paper)}\end{subfigure}&
\begin{subfigure}{0.32\textwidth}\centering\includegraphics[width=0.98\columnwidth, height=4cm]{taxi_3.png}\caption{Boltz (paper)}\end{subfigure}\\
\end{tabular}
\caption{Taxi domain result}
\end{figure}}

% ###########################################################################
% ###########################################################################


\subsection{Lunar Lander Domain}
\subsection*{Experiment Results}
The paper's results:\\
\begin{center}
    \includegraphics[scale=0.7]{lunar_paper1.png}\\
    \includegraphics[scale=0.7]{lunar_paper2.png}
\end{center}
In the paper's results, the Boltzmann softsmax method performs well when $\beta=2, 3$, and the Mellowmax method 
performs well when $\omega=7, 8$.The Boltzmann softmax method, $\beta=10$ performs the worst in the 10 experiments, it 
gets a much lower average mean return than others.The Mellowmax method, $\omega=8$ performs the best in the 10 experiments.\\\\
Our results:\\
\begin{center}
    \includegraphics[scale=0.4]{lunar_rep1.png}
    \includegraphics[scale=0.4]{lunar_rep2.png}\\
    \includegraphics[scale=0.6]{lunar_rep3.png}
\end{center}
In our results, the Boltzmann softmax method performs well when $\beta=3, 5$, and the Mellowmax method 
performs well when $\omega=7, 11$. The Boltzmann softmax method, $\beta=10$ performs the worst in the 10 experiments and
gets a much lower average mean return than others. The Mellowmax method, $\omega=7$ performs the best in our 10 experiments.\\\\
\subsection*{Comparison}
Our experiment results are a little different from the paper's, and we think it is because that we use much less runs and less episodes in each run.
The mean return of Boltzmann softmax $\beta=10$ is about -300, and it's much lower than the paper's result. We find that it gets a lot of negative
 rewards at seed=22, its ewma reward is less than -1000 during almost all episodes, however, its ewma reward reaches 200 while seed=7654.
It seems that Boltzmann softmax $\beta=10$ is unstable, so the chosen seed affects a lots, and more runs are needed to get precise results.
